\documentclass{article} %basic LaTeX document type

%set capital Roman numeral section headings
%set capital Aramaic letters subsection headings
%set capital Arabic numbers subsubsection headings
\renewcommand\thesection{\Roman{section}.}
\renewcommand\thesubsection{\thesection\Alph{subsection}.}
\renewcommand\thesubsubsection{\thesubsection\arabic{subsubsection}.}

%set capital Roman numeral table numeration
\renewcommand*\thetable{\Roman{table}} 

%package needed for next lines
%makes section headings bold and upper case characters
%makes subsubsection headings in italics
\usepackage[explicit]{titlesec}
\titleformat{\section}{\bfseries}{\thesection}{1em}{\MakeUppercase{#1}}
\titleformat{\subsubsection}{\itshape}{\thesubsubsection}{1em}{#1}


%\linespread{2}       %option 1 for making text double-spaced
\usepackage{setspace} %option 2 for making text double-spaced
\doublespacing

%makes first paragraph of section indented (non-first are by default)
%set size of indentation (15pt is default)
\usepackage{indentfirst}
\setlength{\parindent}{25pt}

%set size of all margins
\usepackage[margin=1.3in]{geometry}
%can set margin sizes which are not the same in this way
%\usepackage[left=1in, top=1in, right=1in, bottom=1in]{geometry}

%package which returns number of last page (same as number of pages)
%package which counts the number of tables and/or figures
\usepackage{lastpage}
\usepackage[figure,table]{totalcount}

%enable `align' equation types
%enable `multirow' capability in tables
%enable figures
\usepackage{amsmath}
\usepackage{multirow}
\usepackage{graphicx}

%enables double spaced footnotes
\usepackage[]{footmisc}

%enables subfigures
%enables subfigure captions
%sets table caption formatting options to meet NSE requirements
%sets figure caption options to meet NSE requirements
\usepackage{caption}
\usepackage[labelformat=simple]{subcaption}
\captionsetup[table]{labelsep=newline,name=TABLE}
\captionsetup[figure]{name=Fig.,labelsep=period}

%sets labeling of footnotes
%double spacing of footnotes
\renewcommand{\thefootnote}{\alph{footnote}}
\renewcommand{\footnotelayout}{\doublespacing}

%enables proper labeling of subfigures
\renewcommand*\thesubfigure{(\alph{subfigure})}

%--------------
\usepackage{paralist}	
\usepackage{amssymb}
\usepackage{epsfig}
\usepackage[mathcal]{euscript}
\usepackage{setspace}
\usepackage{color}
\usepackage{array}
%\usepackage{subfigure}
\renewcommand{\ttdefault}{cmtt}
% The float package HAS to load before hyperref
\usepackage{float} % for psuedocode formatting
\usepackage{xspace}
\usepackage{mathrsfs}
\usepackage[pdftex]{hyperref}

%-------------
\DeclareMathOperator{\diag}{diag}
\DeclareMathOperator{\low}{lower}
\DeclareMathOperator{\upp}{upper}

\newcommand{\Sn}{\ensuremath{S_N}}
\newcommand{\Macro}{\ensuremath{\Sigma}}

\newcommand{\vOmega}{\ensuremath{\hat{\Omega}}}
\newcommand{\Ye}[2]{\ensuremath{Y^e_{#1}(\vOmega_#2)}}
\newcommand{\Yo}[2]{\ensuremath{Y^o_{#1}(\vOmega_#2)}}

\newcommand{\ve}[1]{\ensuremath{\mathbf{#1}}}

\newcommand{\sigg}[1]{\ensuremath{\sigma^{gg'}_{\text{s}\,#1}}}
\newcommand{\psig}{\ensuremath{\psi^g}}

\newcommand{\even}{\ensuremath{\phi^g}}
\newcommand{\odd}{\ensuremath{\vartheta^g}}

\newcommand{\evenp}{\ensuremath{\phi^{g'}}}
\newcommand{\oddp}{\ensuremath{\vartheta^{g'}}}

\newcommand{\apsi}[1]{\ensuremath{\psi^{\dagger\,#1}}}
\newcommand{\aeven}[1]{\ensuremath{\phi^{\dagger\,#1}}}
\newcommand{\aodd}[1]{\ensuremath{\vartheta^{\dagger\,#1}}}

\newcommand{\asigg}[1]{\ensuremath{\sigma^{g'g}_{\text{s}\,#1}}}

\newcommand{\aPsi}[1]{\ensuremath{\Psi^{\dagger\,#1}}}
\newcommand{\aPhi}[1]{\ensuremath{\Phi^{\dagger\,#1}}}

\newcommand{\epsi}{\ensuremath{\epsilon}}
\newcommand{\ephi}{\ensuremath{\varepsilon}}

\newcommand{\psie}{\ensuremath{\psi_{\epsi}}}
\newcommand{\phie}{\ensuremath{\phi_{\ephi}}}

\newcommand{\Psie}{\ensuremath{\Psi_{\epsi}}}
\newcommand{\Phie}{\ensuremath{\Phi_{\ephi}}}

\newcommand{\apsie}[1]{\ensuremath{\psi^{\dagger\,#1}_{\epsi}}}
\newcommand{\aphie}[1]{\ensuremath{\phi^{\dagger\,#1}_{\ephi}}}

\newcommand{\aPsie}[1]{\ensuremath{\Psi^{\dagger\,#1}_{\epsi}}}
\newcommand{\aPhie}[1]{\ensuremath{\Phi^{\dagger\,#1}_{\ephi}}}

\newcommand{\avg}[1]{\ensuremath{\langle#1\rangle}}

\newcommand{\osig}{\ensuremath{\overline{\sigma}}}
\newcommand{\osigs}{\ensuremath{\overline{\sigma_{\text{s}}}}}

\begin{document}


%Define fields for \maketitle  (fields are \author, \date, \thanks, and \title)

\title{Evaluation of the Lagrange Discrete Ordinates Equations for Three-Dimensional Neutron Transport} %title of paper

% Potential other title choice is "Assessment of the..." since evaluation could be confused with a mathematical term. I don't have a strong feeling about it though.

\author{
\vspace{20mm}
%list of authors, with corresponding author marked by asterisk
\\Kelly L.\ Rowland,$^{\text{a}}$  Cory D.\ Ahrens,$^\text{b}$ Steven Hamilton,$^\text{c}$ 
\\and R.N.\ Slaybaugh$^{\text{a},\ast}$\\[4pt] 
%affiliations of authors
\textit{$^a$University of California, Berkeley, Nuclear Engineering Department}\\[-10pt]
\textit{4173 Etcheverry Hall, Berkeley, CA 94720, USA} \\[-5pt]
\textit{$^b$X Theoretical Design Division, Primary Physics Group}\\[-10pt]
\textit{Los Alamos National Laboratory, Los Alamos, NM 87545, USA}\\[-5pt]
\textit{$^c$Oak Ridge National Laboratory, Radiation Transport and Criticality Group} \\ [-10pt]
\textit{P.O. Box 2008, Oak Ridge, TN 37831-6170, USA} \\ [-2pt]
{$^\ast$slaybaugh@berkeley.edu}}       %address and email address for correspondence

%instead of returning the date, this repurposes the \maketitle command to print the number of pages, tables, and figures
\date{
\vspace{40mm}
Number of pages: \pageref{LastPage} \\  
Number of tables: \totaltables \\
Number of figures: \totalfigures \\}                                                                                           

\maketitle

\pagebreak

\begin{abstract}
{
abstract

Keywords: x; y; z
}
\end{abstract}

\pagebreak

%%---------------------------------------------------------------------------%%

\section{Introduction}
\label{sec:intro}

%%---------------------------------------------------------------------------%%
%%---------------------------------------------------------------------------%%
%%---------------------------------------------------------------------------%%
\section{Background}
\label{sec:background}

%%---------------------------------------------------------------------------%%
%%---------------------------------------------------------------------------%%
\subsection{LDO Equations}

%%---------------------------------------------------------------------------%%
%%---------------------------------------------------------------------------%%
%%---------------------------------------------------------------------------%%
\section{Past Work}
\label{sec:pastwork}

%%---------------------------------------------------------------------------%%
\section{Results}
\label{sec:results}

%---------------------------------------------------------------------------%%
\section{Conclusions}
\label{sec:conclusions}

\pagebreak
\section*{Acknowledgments}

This material is based upon work supported under an Integrated
University Program Graduate Fellowship as well as supported by the Department 
of Energy under Award Number(s) DE-NE0008661. This report was prepared as an
account of work sponsored by an agency of the United States Government.
Neither the United States Government nor any agency thereof, nor any of their
employees, makes any warranty, express or limited, or assumes any legal
liability or responsibility for the accuracy, completeness, or usefulness of
any information, apparatus, product, or process disclosed, or represents that
its use would not infringe privately owned rights. Reference herein to any 
specific commercial product, process, or service by trade name, trademark, 
manufacturer, or otherwise does not necessarily constitute or imply its 
endorsement, recommendation, or favoring by the United States Government or
any agency thereof. The views and opinions of authors expressed herein do not 
necessarily state or reflect those of the United States Government or any 
agency thereof.

\pagebreak

\bibliographystyle{nse}
\bibliography{ldo-deterministic}

\end{document}

